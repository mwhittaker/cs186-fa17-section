\documentclass{article}

\usepackage[margin=0.75in]{geometry}
\usepackage{etoolbox}
\usepackage{mathtools}
\usepackage{multicol}
\usepackage{multirow}
\usepackage{tikz}
\usepackage{xcolor}
\usetikzlibrary{shapes}

\DeclarePairedDelimiter{\ceil}{\lceil}{\rceil}
\DeclarePairedDelimiter{\floor}{\lfloor}{\rfloor}
\DeclarePairedDelimiter{\parens}{(}{)}
\DeclarePairedDelimiter{\set}{\{}{\}}

\newtoggle{showsolutions}
\toggletrue{showsolutions}
% \togglefalse{showsolutions}
\newcommand{\sol}[1]{\iftoggle{showsolutions}{\textcolor{red}{#1}}{\phantom{#1}}}

\begin{document}
\begin{center}
  \huge Selinger Query Optimization
\end{center}

\section*{Overview}
Consider the relations $R(a, b)$, $S(b, c)$, and $T(c, d)$ with clustered
indexes on $R.a$, $S.b$, and $T.c$ and with unclustered indexes on $S.c$ and
$T.d$. All columns have type \texttt{real}, and all indexes have index keys in
the range $[1, 100]$. We want to optimize the following query:
\begin{verbatim}
  SELECT *
   FROM  R, S, T
  WHERE  R.b = S.b AND S.c = T.c
    AND  R.a <= 50
    AND  (T.c <= 50 AND T.d <= 20)
\end{verbatim}

\section*{Pass 1}
\newcommand{\fts}{FTS}
\newcommand{\is}[1]{IS$_{#1}$}
\newcommand{\select}[1]{$\sigma_{#1}$}

In the first pass of the Selinger query optimization algorithm, we find the
minimum (estimated) cost query plan for every (relation, interesting order)
pair. To do so, we estimate the cost of every possible single-relation query
plan. We can draw each query plans as a tree where \fts{} represents a full
table scan, \is{x} represents a full index scan on a column $x$, and \is{x \leq
k} represents an index scan on column $x$ which also applies the filter $x \leq
k$. We also make sure to push down as many selections as possible. Here are all
eight of the single-relation query plans (which we've also given shorter names
$q_R$, $q_{Ra}$, etc.):

\newcommand{\plan}[1]{\tikz[yscale=0.5]{#1}}

\newcommand{\rfts}{\plan{
  \node{\select{R.a \leq 50}} child {node{\fts} child {node {$R$}}};
}}
\newcommand{\risa}{\plan{ \node{\is{a \leq 50}} child {node{$R$}}; }}
\newcommand{\sfts}{\plan{ \node{\fts} child {node{$S$}}; }}
\newcommand{\sisb}{\plan{ \node{\is{b}} child {node{$S$}}; }}
\newcommand{\sisc}{\plan{ \node{\is{c}} child {node{$S$}}; }}
\newcommand{\tfts}{\plan{
  \node{\select{T.d \leq 20}}
    child { node{\select{T.c \leq 50}}
      child {node{\fts} child {node {$R$}}}
    };
}}
\newcommand{\tisc}{\plan{
  \node{\select{R.d \leq 20}} child {node{\is{c \leq 50}} child {node {$R$}}};
}}
\newcommand{\tisd}{\plan{
  \node{\select{R.c \leq 50}} child {node{\is{d \leq 20}} child {node {$R$}}};
}}

\begin{center}
  \begin{tabular}{|c|c|c|c|c|c|c|c|}
    \hline
    $q_{R}$ & $q_{Ra}$ & $q_{S}$ & $q_{Sb}$ & $q_{Sc}$ & $q_{T}$ & $q_{Tc}$ & $q_{Td}$ \\\hline
    \rfts & \risa & \sfts & \sisb & \sisc & \tfts & \tisc & \tisd \\\hline
  \end{tabular}
\end{center}

Now, let's compute the interesting order, I/O cost, and output size of each of
these query plans.  Let $[R]$, $[S]$, and $[T]$ be the number of \emph{pages}
in $R$, $S$, and $T$. Similarly, let $|R|$, $|S|$, and $|T|$ be the number of
\emph{tuples} in $R$, $S$, and $T$. Also assume it takes $2$ I/Os to traverse
a B+ tree index.

\begin{center}
  \begin{tabular}{|l|l|l|l|}
    \hline
    Query Plan & Interesting Order & I/O Cost     & Output Size \\\hline\hline
    $q_{R}$    & None              & $[R]$        & $0.5[R]$ \\\hline
    $q_{Ra}$   & None              & $2 + 0.5[R]$ & $0.5[R]$ \\\hline
    $q_{S}$    & None              & $[S]$        & $[S]$ \\\hline
    $q_{Sb}$   & $(b)$             & $2 + [S]$    & $[S]$ \\\hline
    $q_{Sc}$   & $(c)$             & $2 + |S|$    & $[S]$ \\\hline
    $q_{T}$    & None              & $[T]$        & $0.1[T]$ \\\hline
    $q_{Tc}$   & $(c)$             & $2 + 0.5[T]$ & $0.1[T]$ \\\hline
    $q_{Td}$   & None              & $2 + 0.2|T|$ & $0.1[T]$ \\\hline
  \end{tabular}
\end{center}

Some notes on interesting order:
\begin{itemize}
  \item
    $q_R$, $q_S$, and $q_T$ have no interesting order because their output is
    not sorted.
  \item
    $q_{Ra}$ produces tuples sorted by $a$, so you might think that the
    interesting order of $q_{Ra}$ is $(a)$. However, the query never joins on
    $R.a$, so we do not consider this sort order interesting. Similarly,
    $q_{Td}$ has no interesting order because the query does not join on $d$.
  \item
    $q_{Sb}$, $q_{Sc}$, and $q_{Tc}$ have interesting orders $(b)$, $(c)$, and
    $(c)$ because their output is sorted on a column which is later joined.
\end{itemize}

Some notes on I/O cost:
\begin{itemize}
  \item
    The I/O cost of $q_{R}$, $q_{S}$, and $q_{T}$ is $[R]$, $[S]$, and $[T]$
    because a full table scan requires us to read every page in a relation.
  \item
    The I/O cost of $q_{Ra}$ is $2 + 0.5[R]$. It takes $2$ I/Os to read the B+
    tree index on $R.a$. Then, we have to read every tuple with $R.a \leq 50$.
    Because the values of $a$ fall in the range $[0, 100]$, the selectivity of
    $R.a \leq 50$ is $0.5$. Thus, we estimate that only half of the tuples
    satisfy $R.a \leq 50$. Because the index is clustered, we only have to read
    half the pages: $0.5[R]$. A similar line of reasoning can be used to
    compute the I/O cost of $q_Sb$ and $q_Tc$.

  \item
    The I/O cost of $q_{Td}$ is $2 + 0.2|T|$. It takes $2$ I/Os to read the B+
    tree index on $T.d$ Then, we have to read every tuple with $T.d \leq 20$.
    Because the values of $d$ fall in the range $[0, 100]$, the selectivity of
    $T.d \leq 50$ is $0.2$. Thus, we estimate that only a fifth of the tuples
    satisfy $T.d \leq 20$. Because the index is unclustered, it takes us $1$
    I/O to retrieve each of these tuples. Thus, it takes $0.2|T|$ I/Os.  A
    similar line of reasoning can be used to compute the I/O cost of $q_{Sc}$.
\end{itemize}

Some notes on output size:
\begin{itemize}
  \item
    Note that the estimated output size of $q_R$ and $q_{Ra}$ is the same; the
    estimated output size of $q_S$, $q_{Sb}$, and $q_{Sc}$ is the same; and the
    estimated output size of $q_T$, $q_{Tc}$, and $q_{Td}$ is the same. This is
    no coincidence. The estimated output size of a relational algebra
    expression will be the same no matter which query plan we choose to
    implement it.

  \item
    The estimated output size of $q_R$ and $q_{Ra}$ is $0.5[R]$ because the
    selectivity of $R.a \leq 50$ is $0.5$.

  \item
    The estimated output size of $q_S$, $q_{Sb}$, and $q_{Sc}$ is $[S]$ because
    we read every single tuple of $S$.

  \item
    The estimated output size of $q_T$, $q_{Tc}$, and $q_{Td}$ is $0.2[T]$
    because the selectivity of $T.c \leq 50 \land T.d \leq 20$ is $0.5 \times
    0.2 = 0.1$.
\end{itemize}

Now that we've computed the I/O cost for every single-relation query plan, we
retain the lowest cost plan for each (relation, interesting order pair).  We'll
use this information in Pass 2:

\begin{center}
  \begin{tabular}{|l|l|l|}
    \hline
    Relation & Interesting Order & Lowest Cost Query Plan \\\hline\hline
    $R$      & None              & $q_{Ra}$ \\\hline
    $S$      & None              & $q_{S}$ \\\hline
    $S$      & $(b)$             & $q_{Sb}$ \\\hline
    $S$      & $(c)$             & $q_{Sc}$ \\\hline
    $T$      & None              & $q_{T}$ \\\hline
    $T$      & $(c)$             & $q_{Tc}$ \\\hline
  \end{tabular}
\end{center}

Notice that we did not retain $q_{R}$ because $q_{Ra}$ had lower cost.
Similarly, we did not retain $q_{Td}$ because $q_{T}$ had lower cost. Also note
that even though $q_{Sc}$ has a very high cost, there is no other
single-relation query plan over $S$ with the interesting order $(c)$, we retain
$q_{Sc}$.

\section*{Pass 2}
\newcommand{\bnlj}[2]{#1 + \ceil*{\frac{#1}{B-2}}#2}
\newcommand{\sortcost}[1]{2#1\ceil*{1 + \log_{B-1}\parens*{\ceil*{\frac{#1}{B}}}}}
\newcommand{\smj}[2]{\text{SC}(#1) + \text{SC}(#2) + #1 + #2}

In the second pass of the Selinger query optimization algorithm, we find the
minimum (estimated) cost query plan for every (set of 2 relation, interesting)
order pair. To do so, we  estimate the cost of every possible two-relation
query plan in which the left and right subqueries are two of the minimum cost
single-relation query plans we computed during Pass 1. Considering only block
nested loop joins (BNLJ) and sort-merge joins (SMJ), we can list all the
two-relation query plans we will consider:

\newcommand{\joinplan}[3]{
  \plan{\node{#1} child {node{$q_{#2}$}} child {node{$q_{#3}$}};}
  \hspace{0.1cm}
}
\newcommand{\bnljplan}[2]{\joinplan{BNLJ}{#1}{#2}}
\newcommand{\smjplan}[2]{\joinplan{SMJ}{#1}{#2}}

\begin{center}
    \bnljplan{Ra}{S}
    \smjplan {Ra}{S}
    \bnljplan{Ra}{Sb}
    \smjplan {Ra}{Sb}
    \bnljplan{Ra}{Sc}
    \smjplan {Ra}{Sc}

    \bnljplan{S }{R}
    \smjplan {S }{R}
    \bnljplan{S }{T}
    \smjplan {S }{T}
    \bnljplan{S }{Tc}
    \smjplan {S }{Tc}

    \bnljplan{Sb}{R}
    \smjplan {Sb}{R}
    \bnljplan{Sb}{T}
    \smjplan {Sb}{T}
    \bnljplan{Sb}{Tc}
    \smjplan {Sb}{Tc}

    \bnljplan{Sc}{R}
    \smjplan {Sc}{R}
    \bnljplan{Sc}{T}
    \smjplan {Sc}{T}
    \bnljplan{Sc}{Tc}
    \smjplan {Sc}{Tc}

    \bnljplan{T }{S}
    \smjplan {T }{S}
    \bnljplan{T }{Sb}
    \smjplan {T }{Sb}
    \bnljplan{T }{Sc}
    \smjplan {T }{Sc}

    \bnljplan{Tc}{S}
    \smjplan {Tc}{S}
    \bnljplan{Tc}{Sb}
    \smjplan {Tc}{Sb}
    \bnljplan{Tc}{Sc}
    \smjplan {Tc}{Sc}
\end{center}

Notice that this list doesn't include \emph{every} pair of single-relation
query plans from Part 1. Notice that BNLJ$(q_{Ra}, q_T)$ is missing for
example. This is because $R$ does not join with $T$ in our query, so we ignore
it. Also notice that BNLJ$(q_{Ra}, q_S)$ is considered but BNLJ$(q_R, q_S)$ is
not. This is because we $q_{Ra}$ has a lower cost than $q_{R}$ and both have
the same interesting order.

As with Pass 1, we now compute the interesting order, I/O cost, and output size
of these query plans. There's a lot of query plans, so let's only look at a
handful. Let $B$ be the number of buffer pages, let $SC(x) = \sortcost{x}$ be
the cost of sorting $x$ pages, and assume the merge phase of a sort merge join
on two relations of size $n$ and $m$ can be done in $n + m$ I/Os.

\begin{center}
  \renewcommand{\arraystretch}{2}
  \begin{tabular}{|l|l|l|l|}
    \hline
    Query Plan             & Interesting Order & I/O Cost                              & Output Size \\\hline\hline
    BNLJ$(q_{Ra}, q_{S})$  & None              & $\bnlj{0.5[R]}{[S]}$                  & $\frac{0.5[R][S]}{100}$ \\\hline
    SMJ$(q_{Ra}, q_{S})$   & None              & $SC(0.5[R]) + SC([S]) + 0.5[R] + [S]$ & $\frac{0.5[R][S]}{100}$ \\\hline
    BNLJ$(q_{Ra}, q_{Sb})$ & None              & $\bnlj{0.5[R]}{[S]}$                  & $\frac{0.5[R][S]}{100}$ \\\hline
    SMJ$(q_{Ra}, q_{Sb})$  & None              & $SC(0.5[R]) + 0.5[R] + [S]$           & $\frac{0.5[R][S]}{100}$ \\\hline
  \end{tabular}
\end{center}

Some notes on interesting order:
\begin{itemize}
  \item
    BNLJ$(q_{Ra}, q_{S})$ and BNLJ$(q_{Ra}, q_{Sb})$ have no interesting order
    because a BNLJ does not output tuples in any particular order, even if one
    or both of its inputs are sorted.
  \item
    SMJ$(q_{Ra}, q_{S})$ and SMJ$(q_{Ra}, q_{Sb})$ output tuples sorted on
    $(b)$, but the $(b)$ column is no longer interesting since it will no
    longer be used as part of a join.
\end{itemize}

Some notes on I/O cost:
\begin{itemize}
  \item
    Estimating the IO cost of BNLJ$(q_{Ra}, q_{S})$ and BNLJ$(q_{Ra}, q_{Sb})$
    is a straightforward use of the BNLJ cost formula where $q_{Ra}$ has
    $0.5[R]$ pages and $q_{Sb}$ has $[S]$ pages. These values were computed in
    Pass 1.

  \item
    The cost of SMJ$(q_{Ra}, q_{S})$ is $SC(0.5[R]) + SC([S]) + 0.5[R] + [S]$.
    Neither $q_{Ra}$ nor $q_S$ output tuples sorted by $(b)$, so we have to
    sort both. Sorting $q_{Ra}$ takes $SC(0.5[R])$ I/Os and sorting $q_S$ takes
    $SC([S])$ I/Os. Merging the two takes $0.5[R] + [S]$ I/Os.

  \item
    The cost of SMJ$(q_{Ra}, q_{Sb})$ is $SC(0.5[R]) + 0.5[R] + [S]$.  $q_{Ra}$
    is not sorted by $(b)$, but $q_{Sb}$ is. Thus, we have to sort $q_{Ra}$,
    which takes $SC(0.5[R])$ I/Os, but we do not have to sort $q_S$.  Merging
    the two takes $0.5[R] + [S]$ I/Os.
\end{itemize}

Some notes on output size:
\begin{itemize}
  \item
    As we noted in Pass 1, the output size of a relational algebra expression
    is the same no matter which plan we choose to implement it. Thus,
    BNLJ$(q_{Ra}, q_{S})$, SMJ$(q_{Ra}, q_{S})$, BNLJ$(q_{Ra}, q_{Sb})$, and
    SMJ$(q_{Ra}, q_{Sb})$ all have the same output size. The maximum size of $R
    \bowtie_{R.b=S.b} S$ is $[R][S]$. The selectivity of $R.a \leq 50 \land
    R.b=S.b$ is $0.5 \times (\frac{1}{\max(50, 100)})$ which is $0.5 \times
    \frac{1}{100}$.  Thus, the output size is $\frac{0.5[R][S]}{100}$.
\end{itemize}
\end{document}
