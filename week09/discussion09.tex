\documentclass{article}

\usepackage[margin=0.75in]{geometry}
\usepackage{etoolbox}
\usepackage{mathtools}
\usepackage{multicol}
\usepackage{multirow}
\usepackage{tikz}
\usepackage{xcolor}
\usetikzlibrary{shapes}

\DeclarePairedDelimiter{\ceil}{\lceil}{\rceil}
\DeclarePairedDelimiter{\floor}{\lfloor}{\rfloor}
\DeclarePairedDelimiter{\parens}{(}{)}

\newtoggle{showsolutions}
\toggletrue{showsolutions}
% \togglefalse{showsolutions}
\newcommand{\sol}[1]{\iftoggle{showsolutions}{\textcolor{red}{#1}}{\phantom{#1}}}

\begin{document}
\begin{center}
  \huge
  CS 186 Discussion 9: \\
  Query Optimization and Database Design
\end{center}

\section{Selectivity Estimation}
Consider a relation $R(a, b, c)$ with 1000 tuples. We have an index on $a$ with
50 unique values in the range $[1, 50]$ and an index on $b$ with 100 unique
values in the range $[1, 100]$. We do not have an index on $c$. Use selectivity
estimation to estimate the number of tuples produced by the following queries.

\begin{enumerate}
  \item \texttt{SELECT * FROM R}
    \sol{1000}
  \item \texttt{SELECT * FROM R WHERE a = 42}
    \sol{$\frac{1}{50} \cdot 1000 = 20$}
  \item \texttt{SELECT * FROM R WHERE b = 42}
    \sol{$\frac{1}{100} \cdot 1000 = 10$}
  \item \texttt{SELECT * FROM R WHERE c = 42}
    \sol{$\frac{1}{10} \cdot 1000 = 100$}
  \item \texttt{SELECT * FROM R WHERE a <= 25}
    \sol{$\frac{1}{2} \cdot 1000 = 500$}
  \item \texttt{SELECT * FROM R WHERE b <= 25}
    \sol{$\frac{1}{4} \cdot 1000 = 250$}
  \item \texttt{SELECT * FROM R WHERE c <= 25}
    \sol{$\frac{1}{10} \cdot 1000 = 100$}
  \item \texttt{SELECT * FROM R WHERE a <= 25 AND b <= 25}
    \sol{$\frac{1}{2}\cdot\frac{1}{4}\cdot 1000 = 125$}
  \item \texttt{SELECT * FROM R WHERE a <= 25 AND c <= 25}
    \sol{$\frac{1}{2}\cdot\frac{1}{10}\cdot 1000 = 50$}
  \item \texttt{SELECT * FROM R WHERE a <= 25 OR b <= 25}
    \sol{$(\frac{1}{2} + \frac{1}{4} - \frac{1}{2}\cdot\frac{1}{4})\cdot 1000 = 625$}
  \item \texttt{SELECT * FROM R WHERE a = b}
    \sol{$\frac{1}{100} \cdot 1000 = 10$}
  \item \texttt{SELECT * FROM R WHERE a = c}
    \sol{$\frac{1}{50} \cdot 1000 = 20$}
\end{enumerate}

\section{Query Optimization}
Consider the relations $R(a, b)$, $S(b, c)$, and $T(c, d)$ with clustered
indexes on $R.a$, $S.b$, and $T.c$ and with unclustered indexes on $S.c$ and
$T.d$. All indexes have index keys in the range $[1, 100]$. We want to optimize
the following query:
\begin{verbatim}
  SELECT *
   FROM  R, S, T
  WHERE  R.b = S.b AND S.c = T.c
    AND  R.a <= 50
    AND  (T.c <= 50 AND T.d <= 20)
\end{verbatim}

In the first pass of the Sellinger query optimization algorithm, we compute the
minimum cost access method for every (relation, interesting order) pair.
Complete the following table which computes this. Let $[R]$, $[S]$, and $[T]$
be the number of \emph{pages} in $R$, $S$, and $T$. Similarly, let $|R|$,
$|S|$, and $|T|$ be the number of \emph{tuples} in $R$, $S$, and $T$.
\begin{center}
  \begin{tabular}{|l|l|l|l|l|l|}
    \hline
                          & Access            & Interesting & I/O                & Output                          & \\
    Relation              & Method            & Order       & Cost               & Size                            & Retained  \\\hline\hline
    \multirow{2}{*}{$R$}  & Full table scan   & None        & $[R]$              & \multirow{2}*{$0.5[R]$}         & No \\\cline{2-4} \cline{6-6}
                          & Index scan on $a$ & $(a)$       & $2 + 0.5[R]$       &                                 & Yes \\\hline
    \multirow{3}{*}{$S$}  & Full table scan   & \sol{None}  & \sol{$[S]$}        & \multirow{3}{*}{\sol{$[S]$}}    & Yes \\\cline{2-4} \cline{6-6}
                          & Index scan on $b$ & \sol{$(b)$} & \sol{$2 + [S]$}    &                                 & Yes \\\cline{2-4} \cline{6-6}
                          & Index scan on $c$ & \sol{$(c)$} & \sol{$2 + |S|$}    &                                 & Yes \\\hline
    \multirow{3}{*}{$T$}  & Full table scan   & \sol{None}  & \sol{$[T]$}        & \multirow{3}{*}{\sol{$0.1[T]$}} & No \\\cline{2-4} \cline{6-6}
                          & Index scan on $c$ & \sol{$(c)$} & \sol{$2 + 0.5[T]$} &                                 & Yes \\\cline{2-4} \cline{6-6}
                          & Index scan on $d$ & \sol{None}  & \sol{$2 + 0.2|T|$} &                                 & No \\\hline
  \end{tabular}
\end{center}

\newcommand{\bnlj}[2]{#1 + \ceil*{\frac{#1}{B-2}}#2}
\newcommand{\sortcost}[1]{2#1\ceil*{1 + \log_{B-1}\parens*{\ceil*{\frac{#1}{B}}}}}
\newcommand{\smj}[2]{\text{SC}(#1) + \text{SC}(#2) + #1 + #2}

In the second pass of the Sellinger query optimization algorithm, we consider
each (relation, interesting order) pair in turn. For each, we compute the cost
of joining every other relation into it. In the end, we retain the minimum cost
join for each (relations, interesting order) pair. Complete the following table
which computes \emph{part} of the second pass. Let $B$ be the number of buffer
pages, let $SC(x) = \sortcost{x}$, assume that a B+ tree index can be traversed
in 2 IOs, and assume the merge phase of a sort merge join on two relations of
size $n$ and $m$ can be done in $n + m$ IOs. Note that we do not consider
joining $R$ and $T$ because we favor joins over cross products.
\begin{center}
  \begin{tabular}{|l|l|l|l|l|l|l|}
    \hline
                         & Left                   &                      &      &                &                                       & \\
    Left                 & Interesting            & Right                &      & Interesting    & I/O                                   & Output                                   \\
    Relations            & Order                  & Relation             & Join & Order          & Cost                                  & Size                                     \\\hline
    \multirow{2}{*}{$R$} & \multirow{2}{*}{$(a)$} & \multirow{2}{*}{$S$} & BNLJ & None           & $\bnlj{0.5[R]}{[S]}$                  & \multirow{2}{*}{$\frac{0.5[R][S]}{100}$} \\\cline{4-6}
                         &                        &                      & SMJ  & $(b)$          & $\text{SC}(0.5[R]) + 0.5[R] + [S]$    & \\\hline
    \multirow{2}{*}{$S$} & \multirow{2}{*}{None}  & \multirow{2}{*}{$R$} & BNLJ & None           & $\bnlj{[S]}{0.5[R]}$                  & \multirow{2}{*}{$\frac{0.5[R][S]}{100}$} \\\cline{4-6}
                         &                        &                      & SMJ  & $(b)$          & $\smj{[S]}{0.5[R]}$                   & \\\hline
    \multirow{2}{*}{$S$} & \multirow{2}{*}{None}  & \multirow{2}{*}{$T$} & BNLJ & \sol{None}     & \sol{$\bnlj{[S]}{0.1[T]}$}            & \multirow{2}{*}{$\frac{0.1[S][T]}{100}$} \\\cline{4-6}
                         &                        &                      & SMJ  & \sol{$(c)$}    & \sol{$\text{SC}([S]) + [S] + 0.1[T]$} & \\\hline
    \multirow{2}{*}{$S$} & \multirow{2}{*}{$(b)$} & \multirow{2}{*}{$R$} & BNLJ & \sol{None}     & \sol{$\bnlj{[S]}{0.5[R]}$}            & \multirow{2}{*}{$\frac{0.5[R][S]}{100}$} \\\cline{4-6}
                         &                        &                      & SMJ  & \sol{$(b)$}    & \sol{$SC(0.5[R]) + [S] + 0.5[R]$}     & \\\hline
    \multirow{2}{*}{$S$} & \multirow{2}{*}{$(b)$} & \multirow{2}{*}{$T$} & BNLJ & \sol{None}     & \sol{$\bnlj{[S]}{0.1[T]}$}            & \multirow{2}{*}{$\frac{0.1[S][T]}{100}$} \\\cline{4-6}
                         &                        &                      & SMJ  & \sol{$(c, b)$} & \sol{$\text{SC}([S]) + [S] + 0.1[T]$} & \\\hline
  \end{tabular}
\end{center}

\section{ER Diagrams}
\newcommand{\basedrawing}{
  \node[draw] (R) at (0, 0) {$R$};
  \node[circle, draw] (Ra) at (-0.5, 1) {\underline{$a$}};
  \node[circle, draw] (Rb) at (0.5, 1) {$b$};

  \node[diamond, draw] (L) at (1.5, 0) {$L$};

  \node[draw] (S) at (3, 0) {$S$};
  \node[circle, draw] (Sa) at (2.5, 1) {\underline{$b$}};
  \node[circle, draw] (Sb) at (3.5, 1) {$c$};
}

Consider the following ER diagrams. We can translate each into an equivalent
triplet of \texttt{CREATE TABLE} statements: one for $R$, one for $S$, and one
for $T$. The statements for $R$ and $S$ remain constant:
\begin{verbatim}
  CREATE TABLE R (a INTEGER, b INTEGER, PRIMARY KEY (a))
  CREATE TABLE S (b INTEGER, c INTEGER, PRIMARY KEY (b))
\end{verbatim}
Provide the \texttt{CREATE TABLE} statements for $L$.
\begin{multicols}{2}
\begin{enumerate}
  \item \hfill

    \begin{tikzpicture}
      \basedrawing
      \draw[ultra thick] (R) to (L);
      \draw[ultra thick] (S) to (L);
    \end{tikzpicture}
{
\iftoggle{showsolutions}{\color{red}}{\color{white}}
\begin{verbatim}
  CREATE TABLE L(
    a INTEGER, b INTEGER,
    FOREIGN KEY (a) REFERENCES R,
    FOREIGN KEY (b) REFERENCES S
  )
\end{verbatim}
}

  \item \hfill

    \begin{tikzpicture}
      \basedrawing
      \draw[ultra thick, ->] (R) to (L);
      \draw[ultra thick] (S) to (L);
    \end{tikzpicture}
{
\iftoggle{showsolutions}{\color{red}}{\color{white}}
\begin{verbatim}
  CREATE TABLE L(
    a INTEGER, b INTEGER,
    PRIMARY KEY (a)
    FOREIGN KEY (a) REFERENCES R,
    FOREIGN KEY (b) REFERENCES S
  )
\end{verbatim}
}

  \item \hfill

    \begin{tikzpicture}
      \basedrawing
      \draw[ultra thick] (R) to (L);
      \draw[ultra thick, ->] (S) to (L);
    \end{tikzpicture}
{
\iftoggle{showsolutions}{\color{red}}{\color{white}}
\begin{verbatim}
  CREATE TABLE L(
    a INTEGER, b INTEGER,
    PRIMARY KEY (b),
    FOREIGN KEY (a) REFERENCES R,
    FOREIGN KEY (b) REFERENCES S
  )
\end{verbatim}
}

\vspace{4cm}\phantom{a}
\end{enumerate}
\end{multicols}
\end{document}
